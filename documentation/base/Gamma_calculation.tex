\documentclass[11pt,a4paper]{article}
\setlength{\parindent}{0em} 
\usepackage[ngerman]{babel}	        % Worttrennung nach deutschen Standrats
\usepackage[utf8]{inputenc}         % umlaute
\usepackage[T1]{fontenc}	        % deutsche Schriftzeichen
\usepackage{longtable}
\usepackage{amsmath}
\usepackage{setspace}
\usepackage{pgfplots}               % for .csv plot
\usepackage{hyperref}
\usepackage{geometry}
\geometry{left=2.5cm, top=2cm, bottom=2cm, right=2cm}

\title{A brief comment of my gamma correction}
\author{Johannes Rothe} 


\begin{document}
\maketitle 
\section{Explanation}
The basic idea: \url{https://en.wikipedia.org/wiki/Gamma_correction}\\
The equation \ref{eq:bit_out} shows how the basic calculation of $bit_{out}$.
The logarithmic function can be vertical adjust with the bit $bit_{out,threshold}$ (equation \ref{eq:bit_out_threshold}).
Therefor threshold parameters in LED drivers can be compensated.

\begin{align} 
    \label{eq:bit_out}
    bit_{out} &= \left(\dfrac{bit_{in}}{bit_{out,max}} \right)^{gamma} \cdot 
        bit_{out,max} \\
    \label{eq:bit_out_threshold}
    bit_{out} &= \left( \left(\dfrac{bit_{in}}{bit_{out,max}} \right)^{gamma} \cdot 
        (bit_{out,max} - bit_{out,threshold}) \right) + bit_{out,threshold}
\end{align}

\indent $bit_{out}$: Output value (e.g. PWM output to driver)\\
\indent $bit_{out,max}$: Output maximum value (e.g. 8-bit MCU = 255) \\
\indent $bit_{in}$: Input value, $bit_{in,max} = bit_{out,max}$ \\
\indent $gamma$: Gamma factor (1,5 to 3)\\
\indent $bit_{out,threshold}$: Value to adjust the output offset \\

\subsection{Example}
An example for a 8-bit MCU, $gamma = 1.5$ and $bit_{out,threshold} = 18$.\\
\begin{equation}\label{eq:ex}
    bit_{out} = \left( \left(\dfrac{bit_{in}}{255} \right)^{1,5} \cdot 
        (255 - 18) \right) + 18
\end{equation}

\begin{tikzpicture}
    \begin{axis}[
        width=.9\textwidth,
        title=Example for equation \ref{eq:ex},  
        xlabel={Input [bit]},
        ylabel={Output [bit]},
        legend style={at={(0.98,0.3)},anchor=south east},
        grid=both,
        xtick={0,50,100,150,200,250},
        ytick={0,50,100,150,200,250},
        xmin=0, xmax=255,
        ymin=0, ymax=255]
        \addplot[red, ultra thick] table[col sep=comma,x index=0 ,y index=4] {Gamma_Cal.CSV};
        \addlegendentry{Function of equation \ref{eq:ex}}
        \addplot[black!30] table[col sep=comma,x index=0 ,y index=0] {Gamma_Cal.CSV};
        \addlegendentry{Function f(x) = y}
    \end{axis}
\end{tikzpicture}

% \newpage
% % \usepackage{longtable}


\begin{longtable}{r|rrrr}[c]
        & \multicolumn{4}{c|}{Output value} \\
    \hline
    \multicolumn{1}{|l|}{Input value} & \multicolumn{1}{l|}{Gamma 1,5} & \multicolumn{1}{l|}{Gamma 2} & \multicolumn{1}{l|}{Custom Gamma 2} & \multicolumn{1}{l|}{Custom Gamma 1,5} \\
    \hline
    0     & 0     & 0     & 0     & 0 \\
    1     & 0,062622429 & 0,003921569 & 0     & 0 \\
    2     & 0,177122977 & 0,015686275 & 0     & 0 \\
    3     & 0,325395687 & 0,035294118 & 0     & 0 \\
    4     & 0,500979433 & 0,062745098 & 18,05831603 & 18,46561618 \\
    5     & 0,700140042 & 0,098039216 & 18,0911188 & 18,65071839 \\
    6     & 0,920357987 & 0,141176471 & 18,13121107 & 18,85539154 \\
    7     & 1,159783617 & 0,192156863 & 18,17859285 & 19,07791654 \\
    8     & 1,416983817 & 0,250980392 & 18,23326413 & 19,31696143 \\
    9     & 1,690805586 & 0,317647059 & 18,29522491 & 19,5714546 \\
    10    & 1,980295086 & 0,392156863 & 18,3644752 & 19,84050955 \\
    11    & 2,284646109 & 0,474509804 & 18,44101499 & 20,12337697 \\
    12    & 2,603165494 & 0,564705882 & 18,52484429 & 20,41941264 \\
    13    & 2,935248929 & 0,662745098 & 18,61596309 & 20,72805489 \\
    14    & 3,280363442 & 0,768627451 & 18,7143714 & 21,04880838 \\
    15    & 3,638034376 & 0,882352941 & 18,8200692 & 21,38123195 \\
    16    & 4,007835463 & 1,003921569 & 18,93305652 & 21,72492943 \\
    17    & 4,389381126 & 1,133333333 & 19,05333333 & 22,07954246 \\
    18    & 4,782320382 & 1,270588235 & 19,18089965 & 22,44474483 \\
    19    & 5,186331962 & 1,415686275 & 19,31575548 & 22,82023794 \\
    20    & 5,601120336 & 1,568627451 & 19,45790081 & 23,20574714 \\
    21    & 6,026412453 & 1,729411765 & 19,60733564 & 23,60101863 \\
    22    & 6,461955025 & 1,898039216 & 19,76405998 & 24,00581702 \\
    23    & 6,90751225 & 2,074509804 & 19,92807382 & 24,41992315 \\
    24    & 7,362863893 & 2,258823529 & 20,09937716 & 24,84313232 \\
    25    & 7,827803639 & 2,450980392 & 20,27797001 & 25,27525279 \\
    26    & 8,302137688 & 2,650980392 & 20,46385236 & 25,71610444 \\
    27    & 8,785683542 & 2,858823529 & 20,65702422 & 26,16551764 \\
    28    & 9,278268939 & 3,074509804 & 20,85748558 & 26,62333231 \\
    29    & 9,77973094 & 3,298039216 & 21,06523645 & 27,08939699 \\
    30    & 10,28991511 & 3,529411765 & 21,28027682 & 27,56356816 \\
    31    & 10,8086748 & 3,768627451 & 21,50260669 & 28,04570952 \\
    32    & 11,33587053 & 4,015686275 & 21,73222607 & 28,53569144 \\
    33    & 11,87136941 & 4,270588235 & 21,96913495 & 29,0333904 \\
    34    & 12,41504464 & 4,533333333 & 22,21333333 & 29,53868854 \\
    35    & 12,96677504 & 4,803921569 & 22,46482122 & 30,05147327 \\
    36    & 13,52644469 & 5,082352941 & 22,72359862 & 30,57163683 \\
    37    & 14,09394252 & 5,368627451 & 22,98966551 & 31,09907599 \\
    38    & 14,669162 & 5,662745098 & 23,26302191 & 31,63369174 \\
    39    & 15,25200083 & 5,964705882 & 23,54366782 & 32,17538901 \\
    40    & 15,84236069 & 6,274509804 & 23,83160323 & 32,7240764 \\
    41    & 16,44014694 & 6,592156863 & 24,12682814 & 33,27966598 \\
    42    & 17,04526845 & 6,917647059 & 24,42934256 & 33,84207303 \\
    43    & 17,65763735 & 7,250980392 & 24,73914648 & 34,41121589 \\
    44    & 18,27716887 & 7,592156863 & 25,05623991 & 34,98701577 \\
    45    & 18,90378113 & 7,941176471 & 25,38062284 & 35,56939658 \\
    46    & 19,53739501 & 8,298039216 & 25,71229527 & 36,15828478 \\
    47    & 20,17793398 & 8,662745098 & 26,05125721 & 36,75360923 \\
    48    & 20,82532395 & 9,035294118 & 26,39750865 & 37,35530108 \\
    49    & 21,47949318 & 9,415686275 & 26,7510496 & 37,96329367 \\
    50    & 22,14037214 & 9,803921569 & 27,11188005 & 38,57752234 \\
    51    & 22,80789337 & 10,2  & 27,48 & 39,19792443 \\
    52    & 23,48199143 & 10,60392157 & 27,85540946 & 39,82443909 \\
    53    & 24,16260277 & 11,01568627 & 28,23810842 & 40,45700728 \\
    54    & 24,84966564 & 11,43529412 & 28,62809689 & 41,09557159 \\
    55    & 25,54312002 & 11,8627451 & 29,02537486 & 41,74007625 \\
    56    & 26,24290754 & 12,29803922 & 29,42994233 & 42,39046701 \\
    57    & 26,94897139 & 12,74117647 & 29,84179931 & 43,04669105 \\
    58    & 27,66125626 & 13,19215686 & 30,26094579 & 43,708697 \\
    59    & 28,3797083 & 13,65098039 & 30,68738178 & 44,37643477 \\
    60    & 29,104275 & 14,11764706 & 31,12110727 & 45,04985559 \\
    61    & 29,83490521 & 14,59215686 & 31,56212226 & 45,7289119 \\
    62    & 30,57154899 & 15,0745098 & 32,01042676 & 46,4135573 \\
    63    & 31,31415767 & 15,56470588 & 32,46602076 & 47,10374654 \\
    64    & 32,0626837 & 16,0627451 & 32,92890427 & 47,79943544 \\
    65    & 32,81708068 & 16,56862745 & 33,39907728 & 48,50058087 \\
    66    & 33,57730326 & 17,08235294 & 33,87653979 & 49,20714068 \\
    67    & 34,34330714 & 17,60392157 & 34,36129181 & 49,91907369 \\
    68    & 35,11504901 & 18,13333333 & 34,85333333 & 50,63633966 \\
    69    & 35,89248652 & 18,67058824 & 35,35266436 & 51,35889923 \\
    70    & 36,67557824 & 19,21568627 & 35,85928489 & 52,08671389 \\
    71    & 37,46428364 & 19,76862745 & 36,37319493 & 52,81974598 \\
    72    & 38,25856306 & 20,32941176 & 36,89439446 & 53,5579586 \\
    73    & 39,05837763 & 20,89803922 & 37,42288351 & 54,30131568 \\
    74    & 39,86368931 & 21,4745098 & 37,95866205 & 55,04978183 \\
    75    & 40,67446084 & 22,05882353 & 38,5017301 & 55,80332243 \\
    76    & 41,49065569 & 22,65098039 & 39,05208766 & 56,56190353 \\
    77    & 42,31223807 & 23,25098039 & 39,60973472 & 57,32549185 \\
    78    & 43,13917286 & 23,85882353 & 40,17467128 & 58,09405478 \\
    79    & 43,97142566 & 24,4745098 & 40,74689735 & 58,86756032 \\
    80    & 44,80896269 & 25,09803922 & 41,32641292 & 59,64597709 \\
    81    & 45,65175082 & 25,72941176 & 41,91321799 & 60,42927429 \\
    82    & 46,49975754 & 26,36862745 & 42,50731257 & 61,21742171 \\
    83    & 47,35295092 & 27,01568627 & 43,10869666 & 62,01038968 \\
    84    & 48,21129963 & 27,67058824 & 43,71737024 & 62,80814907 \\
    85    & 49,07477288 & 28,33333333 & 44,33333333 & 63,61067127 \\
    86    & 49,94334045 & 29,00392157 & 44,95658593 & 64,41792818 \\
    87    & 50,81697262 & 29,68235294 & 45,58712803 & 65,2298922 \\
    88    & 51,6956402 & 30,36862745 & 46,22495963 & 66,04653619 \\
    89    & 52,5793145 & 31,0627451 & 46,87008074 & 66,86783348 \\
    90    & 53,46796732 & 31,76470588 & 47,52249135 & 67,69375786 \\
    91    & 54,36157091 & 32,4745098 & 48,18219146 & 68,52428355 \\
    92    & 55,260098 & 33,19215686 & 48,84918108 & 69,3593852 \\
    93    & 56,16352176 & 33,91764706 & 49,52346021 & 70,19903787 \\
    94    & 57,07181578 & 34,65098039 & 50,20502884 & 71,04321702 \\
    95    & 57,9849541 & 35,39215686 & 50,89388697 & 71,89189852 \\
    96    & 58,90291114 & 36,14117647 & 51,5900346 & 72,74505859 \\
    97    & 59,82566175 & 36,89803922 & 52,29347174 & 73,60267386 \\
    98    & 60,75318115 & 37,6627451 & 53,00419839 & 74,4647213 \\
    99    & 61,68544494 & 38,43529412 & 53,72221453 & 75,33117824 \\
    100   & 62,62242911 & 39,21568627 & 54,44752018 & 76,20202235 \\
    101   & 63,56410999 & 40,00392157 & 55,18011534 & 77,07723163 \\
    102   & 64,51046427 & 40,8  & 55,92 & 77,95678444 \\
    103   & 65,46146898 & 41,60392157 & 56,66717416 & 78,84065941 \\
    104   & 66,41710151 & 42,41568627 & 57,42163783 & 79,72883552 \\
    105   & 67,37733953 & 43,23529412 & 58,183391 & 80,62129204 \\
    106   & 68,34216107 & 44,0627451 & 58,95243368 & 81,51800853 \\
    107   & 69,31154446 & 44,89803922 & 59,72876586 & 82,41896486 \\
    108   & 70,28546833 & 45,74117647 & 60,51238754 & 83,32414116 \\
    109   & 71,26391161 & 46,59215686 & 61,30329873 & 84,23351785 \\
    110   & 72,24685352 & 47,45098039 & 62,10149942 & 85,14707562 \\
    111   & 73,23427356 & 48,31764706 & 62,90698962 & 86,06479542 \\
    112   & 74,22615151 & 49,19215686 & 63,71976932 & 86,98665847 \\
    113   & 75,22246744 & 50,0745098 & 64,53983852 & 87,91264621 \\
    114   & 76,22320166 & 50,96470588 & 65,36719723 & 88,84274036 \\
    115   & 77,22833474 & 51,8627451 & 66,20184544 & 89,77692287 \\
    116   & 78,23784752 & 52,76862745 & 67,04378316 & 90,71517593 \\
    117   & 79,25172108 & 53,68235294 & 67,89301038 & 91,65748194 \\
    118   & 80,26993675 & 54,60392157 & 68,7495271 & 92,60382356 \\
    119   & 81,29247608 & 55,53333333 & 69,61333333 & 93,55418365 \\
    120   & 82,31932087 & 56,47058824 & 70,48442907 & 94,50854528 \\
    121   & 83,35045314 & 57,41568627 & 71,3628143 & 95,46689175 \\
    122   & 84,38585515 & 58,36862745 & 72,24848904 & 96,42920655 \\
    123   & 85,42550935 & 59,32941176 & 73,14145329 & 97,39547339 \\
    124   & 86,46939842 & 60,29803922 & 74,04170704 & 98,36567618 \\
    125   & 87,51750525 & 61,2745098 & 74,94925029 & 99,339799 \\
    126   & 88,56981294 & 62,25882353 & 75,86408304 & 100,3178261 \\
    127   & 89,62630479 & 63,25098039 & 76,78620531 & 101,2997421 \\
    128   & 90,68696428 & 64,25098039 & 77,71561707 & 102,2855315 \\
    129   & 91,75177511 & 65,25882353 & 78,65231834 & 103,2751792 \\
    130   & 92,82072115 & 66,2745098 & 79,59630911 & 104,2686702 \\
    131   & 93,89378647 & 67,29803922 & 80,54758939 & 105,2659898 \\
    132   & 94,97095531 & 68,32941176 & 81,50615917 & 106,2671232 \\
    133   & 96,05221211 & 69,36862745 & 82,47201845 & 107,272056 \\
    134   & 97,13754146 & 70,41568627 & 83,44516724 & 108,2807738 \\
    135   & 98,22692814 & 71,47058824 & 84,42560554 & 109,2932626 \\
    136   & 99,32035709 & 72,53333333 & 85,41333333 & 110,3095084 \\
    137   & 100,4178134 & 73,60392157 & 86,40835063 & 111,3294972 \\
    138   & 101,5192824 & 74,68235294 & 87,41065744 & 112,3532154 \\
    139   & 102,6247495 & 75,76862745 & 88,42025375 & 113,3806496 \\
    140   & 103,7342003 & 76,8627451 & 89,43713956 & 114,4117862 \\
    141   & 104,8476205 & 77,96470588 & 90,46131488 & 115,446612 \\
    142   & 105,9649961 & 79,0745098 & 91,4927797 & 116,485114 \\
    143   & 107,086313 & 80,19215686 & 92,53153403 & 117,5272791 \\
    144   & 108,2115575 & 81,31764706 & 93,57757785 & 118,5730946 \\
    145   & 109,3407159 & 82,45098039 & 94,63091119 & 119,6225477 \\
    146   & 110,4737747 & 83,59215686 & 95,69153403 & 120,6756259 \\
    147   & 111,6107205 & 84,74117647 & 96,75944637 & 121,7323167 \\
    148   & 112,7515401 & 85,89803922 & 97,83464821 & 122,7926079 \\
    149   & 113,8962204 & 87,0627451 & 98,91713956 & 123,8564872 \\
    150   & 115,0447483 & 88,23529412 & 100,0069204 & 124,9239426 \\
    151   & 116,1971111 & 89,41568627 & 101,1039908 & 125,9949621 \\
    152   & 117,353296 & 90,60392157 & 102,2083506 & 127,0695339 \\
    153   & 118,5132904 & 91,8  & 103,32 & 128,1476464 \\
    154   & 119,6770819 & 93,00392157 & 104,4389389 & 129,2292878 \\
    155   & 120,844658 & 94,21568627 & 105,5651672 & 130,3144469 \\
    156   & 122,0160067 & 95,43529412 & 106,6986851 & 131,4031121 \\
    157   & 123,1911157 & 96,6627451 & 107,8394925 & 132,4952722 \\
    158   & 124,369973 & 97,89803922 & 108,9875894 & 133,5909161 \\
    159   & 125,5525669 & 99,14117647 & 110,1429758 & 134,6900328 \\
    160   & 126,7388855 & 100,3921569 & 111,3056517 & 135,7926112 \\
    161   & 127,9289171 & 101,6509804 & 112,4756171 & 136,8986406 \\
    162   & 129,1226503 & 102,9176471 & 113,652872 & 138,0081103 \\
    163   & 130,3200735 & 104,1921569 & 114,8374164 & 139,1210095 \\
    164   & 131,5211755 & 105,4745098 & 116,0292503 & 140,2373278 \\
    165   & 132,725945 & 106,7647059 & 117,2283737 & 141,3570547 \\
    166   & 133,9343708 & 108,0627451 & 118,4347866 & 142,4801799 \\
    167   & 135,146442 & 109,3686275 & 119,648489 & 143,6066931 \\
    168   & 136,3621476 & 110,6823529 & 120,869481 & 144,7365842 \\
    169   & 137,5814768 & 112,0039216 & 122,0977624 & 145,8698431 \\
    170   & 138,8044188 & 113,3333333 & 123,3333333 & 147,0064598 \\
    171   & 140,030963 & 114,6705882 & 124,5761938 & 148,1464244 \\
    172   & 141,2610988 & 116,0156863 & 125,8263437 & 149,2897271 \\
    173   & 142,4948159 & 117,3686275 & 127,0837832 & 150,4363583 \\
    174   & 143,7321037 & 118,7294118 & 128,3485121 & 151,5863082 \\
    175   & 144,9729522 & 120,0980392 & 129,6205306 & 152,7395673 \\
    176   & 146,217351 & 121,4745098 & 130,8998385 & 153,8961262 \\
    177   & 147,46529 & 122,8588235 & 132,186436 & 155,0559754 \\
    178   & 148,7167593 & 124,2509804 & 133,480323 & 156,2191057 \\
    179   & 149,971749 & 125,6509804 & 134,7814994 & 157,3855079 \\
    180   & 151,2302491 & 127,0588235 & 136,0899654 & 158,5551727 \\
    181   & 152,4922499 & 128,4745098 & 137,4057209 & 159,7280911 \\
    182   & 153,7577417 & 129,8980392 & 138,7287659 & 160,9042541 \\
    183   & 155,026715 & 131,3294118 & 140,0591003 & 162,0836527 \\
    184   & 156,2991601 & 132,7686275 & 141,3967243 & 163,2662782 \\
    185   & 157,5750677 & 134,2156863 & 142,7416378 & 164,4521217 \\
    186   & 158,8544284 & 135,6705882 & 144,0938408 & 165,6411746 \\
    187   & 160,1372328 & 137,1333333 & 145,4533333 & 166,8334281 \\
    188   & 161,4234718 & 138,6039216 & 146,8201153 & 168,0288738 \\
    189   & 162,7131362 & 140,0823529 & 148,1941869 & 169,2275031 \\
    190   & 164,006217 & 141,5686275 & 149,5755479 & 170,4293076 \\
    191   & 165,3027051 & 143,0627451 & 150,9641984 & 171,6342789 \\
    192   & 166,6025916 & 144,5647059 & 152,3601384 & 172,8424087 \\
    193   & 167,9058677 & 146,0745098 & 153,7633679 & 174,0536888 \\
    194   & 169,2125245 & 147,5921569 & 155,173887 & 175,268111 \\
    195   & 170,5225533 & 149,1176471 & 156,5916955 & 176,4856672 \\
    196   & 171,8359455 & 150,6509804 & 158,0167935 & 177,7063493 \\
    197   & 173,1526924 & 152,1921569 & 159,4491811 & 178,9301495 \\
    198   & 174,4727857 & 153,7411765 & 160,8888581 & 180,1570596 \\
    199   & 175,7962167 & 155,2980392 & 162,3358247 & 181,387072 \\
    200   & 177,1229771 & 156,8627451 & 163,7900807 & 182,6201787 \\
    201   & 178,4530586 & 158,4352941 & 165,2516263 & 183,8563721 \\
    202   & 179,7864528 & 160,0156863 & 166,7204614 & 185,0956444 \\
    203   & 181,1231517 & 161,6039216 & 168,1965859 & 186,337988 \\
    204   & 182,463147 & 163,2 & 169,68 & 187,5833954 \\
    205   & 183,8064306 & 164,8039216 & 171,1707036 & 188,831859 \\
    206   & 185,1529945 & 166,4156863 & 172,6686967 & 190,0833714 \\
    207   & 186,5028308 & 168,0352941 & 174,1739792 & 191,3379251 \\
    208   & 187,8559314 & 169,6627451 & 175,6865513 & 192,5955128 \\
    209   & 189,2122887 & 171,2980392 & 177,2064129 & 193,8561271 \\
    210   & 190,5718947 & 172,9411765 & 178,733564 & 195,119761 \\
    211   & 191,9347418 & 174,5921569 & 180,2680046 & 196,3864071 \\
    212   & 193,3008221 & 176,2509804 & 181,8097347 & 197,6560582 \\
    213   & 194,6701282 & 177,9176471 & 183,3587543 & 198,9287074 \\
    214   & 196,0426524 & 179,5921569 & 184,9150634 & 200,2043475 \\
    215   & 197,4183872 & 181,2745098 & 186,4786621 & 201,4829716 \\
    216   & 198,7973251 & 182,9647059 & 188,0495502 & 202,7645727 \\
    217   & 200,1794587 & 184,6627451 & 189,6277278 & 204,049144 \\
    218   & 201,5647806 & 186,3686275 & 191,2131949 & 205,3366785 \\
    219   & 202,9532835 & 188,0823529 & 192,8059516 & 206,6271694 \\
    220   & 204,3449602 & 189,8039216 & 194,4059977 & 207,92061 \\
    221   & 205,7398033 & 191,5333333 & 196,0133333 & 209,2169937 \\
    222   & 207,1378058 & 193,2705882 & 197,6279585 & 210,5163136 \\
    223   & 208,5389605 & 195,0156863 & 199,2498731 & 211,8185633 \\
    224   & 209,9432603 & 196,7686275 & 200,8790773 & 213,1237361 \\
    225   & 211,3506982 & 198,5294118 & 202,5155709 & 214,4318254 \\
    226   & 212,7612673 & 200,2980392 & 204,1593541 & 215,7428249 \\
    227   & 214,1749605 & 202,0745098 & 205,8104268 & 217,056728 \\
    228   & 215,5917711 & 203,8588235 & 207,4687889 & 218,3735284 \\
    229   & 217,0116921 & 205,6509804 & 209,1344406 & 219,6932197 \\
    230   & 218,4347168 & 207,4509804 & 210,8073818 & 221,0157956 \\
    231   & 219,8608383 & 209,2588235 & 212,4876125 & 222,3412498 \\
    232   & 221,2900501 & 211,0745098 & 214,1751326 & 223,669576 \\
    233   & 222,7223454 & 212,8980392 & 215,8699423 & 225,0007681 \\
    234   & 224,1577176 & 214,7294118 & 217,5720415 & 226,3348199 \\
    235   & 225,5961601 & 216,5686275 & 219,2814302 & 227,6717253 \\
    236   & 227,0376664 & 218,4156863 & 220,9981084 & 229,0114782 \\
    237   & 228,48223 & 220,2705882 & 222,7220761 & 230,3540726 \\
    238   & 229,9298444 & 222,1333333 & 224,4533333 & 231,6995024 \\
    239   & 231,3805032 & 224,0039216 & 226,19188 & 233,0477618 \\
    240   & 232,8342 & 225,8823529 & 227,9377163 & 234,3988447 \\
    241   & 234,2909286 & 227,7686275 & 229,690842 & 235,7527454 \\
    242   & 235,7506825 & 229,6627451 & 231,4512572 & 237,1094579 \\
    243   & 237,2134556 & 231,5647059 & 233,2189619 & 238,4689764 \\
    244   & 238,6792416 & 233,4745098 & 234,9939562 & 239,8312952 \\
    245   & 240,1480344 & 235,3921569 & 236,7762399 & 241,1964085 \\
    246   & 241,6198278 & 237,3176471 & 238,5658131 & 242,5643105 \\
    247   & 243,0946156 & 239,2509804 & 240,3626759 & 243,9349957 \\
    248   & 244,5723919 & 241,1921569 & 242,1668281 & 245,3084584 \\
    249   & 246,0531506 & 243,1411765 & 243,9782699 & 246,684693 \\
    250   & 247,5368857 & 245,0980392 & 245,7970012 & 248,0636938 \\
    251   & 249,0235913 & 247,0627451 & 247,6230219 & 249,4454554 \\
    252   & 250,5132614 & 249,0352941 & 249,4563322 & 250,8299723 \\
    253   & 252,0058901 & 251,0156863 & 251,2969319 & 252,217239 \\
    254   & 253,5014716 & 253,0039216 & 253,1448212 & 253,60725 \\
    255   & 255   & 255   & 255   & 255 \\
\end{longtable}    
\end{document}